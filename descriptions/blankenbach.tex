% Copyright (C) 2013 Columbia University in the City of New York and others.
%
% Please see the AUTHORS file in the main source directory for a full list
% of contributors.
%
% This file is part of TerraFERMA.
%
% TerraFERMA is free software: you can redistribute it and/or modify
% it under the terms of the GNU Lesser General Public License as published by
% the Free Software Foundation, either version 3 of the License, or
% (at your option) any later version.
%
% TerraFERMA is distributed in the hope that it will be useful,
% but WITHOUT ANY WARRANTY; without even the implied warranty of
% MERCHANTABILITY or FITNESS FOR A PARTICULAR PURPOSE. See the
% GNU Lesser General Public License for more details.
%
% You should have received a copy of the GNU Lesser General Public License
% along with TerraFERMA. If not, see <http://www.gnu.org/licenses/>.

\chapter{Incompressible Two-Dimensional Convection: \citeauthor{BlankenbachGJI1989}} \label{sec:blankenbach}

\citet{BlankenbachGJI1989} described several benchmarks for
incompressible convection in a two-dimensional domain of unit height
and aspect ratio $l$.  The prognostic variables are velocity, $\vec{v}$,
pressure $p$, and temperature, $T$.

For the steady-state cases (1--2) boundary conditions for temperature are $T=0$ at the top surface, $z=1$, $T=1$ at the base, $z=0$,
with insulating (homogeneous Neumann, $\partial_x T = 0$) side-walls.  For velocity, free-slip boundary conditions are specified at
all boundaries.  Simulations are run at a variety of resolutions, all structured but with refinement at the boundaries, until a near
steady-state is attained where the variation in the fields is less than $10^{-9}$ in the infinity-norm between time-steps.

\paragraph{Case 1a--c}
Isoviscous steady-state cases are defined in a domain with aspect ratio, $l=1$,
with Rayleigh numbers $10^4$, $10^{5}$
and $10^{6}$ (Table \ref{tab:blankenbach1a-c})

\begin{table*}[th!]
\caption{Results from 2D, isoviscous square convection benchmark
  cases \citep{BlankenbachGJI1989}.}
\begin{center}
  \begin{tabular}{ll}
\fbox{\begin{minipage}[c]{0.025\textwidth}\begin{sideways}Case 1a: $\Ra=10^{4}$\end{sideways}\end{minipage}} 
&
\fbox{\begin{minipage}[c]{0.65\textwidth}\begin{tabular}{c|cccccc}
    & $\Nu$ & $v_{rms}$ & $q_1$ & $q_2$ & $T_{e}$ & $z_{e}$ \\
\hline
16$\times$16 & 4.897 & 42.884 & 8.062 & 0.588 & 0.423 & 0.233 \\
32$\times$32 & 4.887 & 42.865 & 8.060 & 0.589 & 0.422 & 0.226 \\
64$\times$64 & 4.885 & 42.865 & 8.059 & 0.589 & 0.422 & 0.226 \\
128$\times$128 & 4.885 & 42.865 & 8.059 & 0.589 & 0.422 & 0.225 \\
256$\times$256 & 4.884 & 42.865 & 8.059 & 0.589 & 0.422 & 0.225 \\
\hline
Benchmark & 4.884 & 42.865 & 8.059 & 0.589 & 0.422 & 0.225 \\
\end{tabular}
\end{minipage}} \\
\\
\fbox{\begin{minipage}[c]{0.025\textwidth}\begin{sideways}Case 1b: $\Ra=10^{5}$\end{sideways}\end{minipage}} 
&
\fbox{\begin{minipage}[c]{0.65\textwidth}\begin{tabular}{c|cccccc}
    & $\Nu$ & $v_{rms}$ & $q_1$ & $q_2$ & $T_{e}$ & $z_{e}$ \\
\hline
16$\times$16 & 4.897 & 42.884 & 8.062 & 0.588 & 0.423 & 0.233 \\
32$\times$32 & 4.887 & 42.865 & 8.060 & 0.589 & 0.422 & 0.226 \\
64$\times$64 & 4.885 & 42.865 & 8.059 & 0.589 & 0.422 & 0.226 \\
128$\times$128 & 4.885 & 42.865 & 8.059 & 0.589 & 0.422 & 0.225 \\
256$\times$256 & 4.884 & 42.865 & 8.059 & 0.589 & 0.422 & 0.225 \\
\hline
Benchmark & 4.884 & 42.865 & 8.059 & 0.589 & 0.422 & 0.225 \\
\end{tabular}
\end{minipage}} \\
\\
\fbox{\begin{minipage}[c]{0.025\textwidth}\begin{sideways}Case 1c: $\Ra=10^{6}$\end{sideways}\end{minipage}} 
&
\fbox{\begin{minipage}[c]{0.65\textwidth}\begin{tabular}{c|cccccc}
    & $\Nu$ & $v_{rms}$ & $q_1$ & $q_2$ & $T_{e}$ & $z_{e}$ \\
\hline
16$\times$16 & 4.897 & 42.884 & 8.062 & 0.588 & 0.423 & 0.233 \\
32$\times$32 & 4.887 & 42.865 & 8.060 & 0.589 & 0.422 & 0.226 \\
64$\times$64 & 4.885 & 42.865 & 8.059 & 0.589 & 0.422 & 0.226 \\
128$\times$128 & 4.885 & 42.865 & 8.059 & 0.589 & 0.422 & 0.225 \\
256$\times$256 & 4.884 & 42.865 & 8.059 & 0.589 & 0.422 & 0.225 \\
\hline
Benchmark & 4.884 & 42.865 & 8.059 & 0.589 & 0.422 & 0.225 \\
\end{tabular}
\end{minipage}}
  \end{tabular}
\end{center}
  \label{tab:blankenbach1a-c}
\end{table*}

\paragraph{Case 2a-b}

Two variable viscosity steady-state cases were run. For case \textbf{2a},
viscosity is temperature dependent with 
\begin{equation}
 \mu = \exp\left(-bT_i\right)
\end{equation}
and $b=\ln(1000)$.
For case \textbf{2b} the viscosity is also depth-dependent according to the equation:
\begin{equation}
 \mu = \exp\left(-bT_i + c(1-z)\right)
\end{equation}
where $b=\ln(16384)$ and $c=\ln(64)$.
Convergence results are given in Table \ref{tab:blankenbach2a-b} for
$\Ra=10^4$, aspect ratio $l=1$.

\begin{table*}[th!]
\caption{Results from 2D, variable viscosity square convection benchmark
  cases $\Ra=10^{4}$ \citep{BlankenbachGJI1989}.}
\begin{center}
  \begin{tabular}{ll}
\fbox{\begin{minipage}[c]{0.025\textwidth}\begin{sideways}Case 2a: $\eta(T)$\end{sideways}\end{minipage}} 
&
\fbox{\begin{minipage}[c]{0.925\textwidth}\begin{tabular}{c|cccccc}
    & $\Nu$ & $v_{rms}$ & $q_1$ & $q_2$ & $T_{e}$ & $z_{e}$ \\
\hline
16$\times$16 & 4.897 & 42.884 & 8.062 & 0.588 & 0.423 & 0.233 \\
32$\times$32 & 4.887 & 42.865 & 8.060 & 0.589 & 0.422 & 0.226 \\
64$\times$64 & 4.885 & 42.865 & 8.059 & 0.589 & 0.422 & 0.226 \\
128$\times$128 & 4.885 & 42.865 & 8.059 & 0.589 & 0.422 & 0.225 \\
256$\times$256 & 4.884 & 42.865 & 8.059 & 0.589 & 0.422 & 0.225 \\
\hline
Benchmark & 4.884 & 42.865 & 8.059 & 0.589 & 0.422 & 0.225 \\
\end{tabular}
\end{minipage}} \\
\\
\fbox{\begin{minipage}[c]{0.025\textwidth}\begin{sideways}Case 2b: $\eta(T,z)$\end{sideways}\end{minipage}} 
&
\fbox{\begin{minipage}[c]{0.925\textwidth}\begin{tabular}{c|cccccc}
    & $\Nu$ & $v_{rms}$ & $q_1$ & $q_2$ & $T_{e}$ & $z_{e}$ \\
\hline
16$\times$16 & 4.897 & 42.884 & 8.062 & 0.588 & 0.423 & 0.233 \\
32$\times$32 & 4.887 & 42.865 & 8.060 & 0.589 & 0.422 & 0.226 \\
64$\times$64 & 4.885 & 42.865 & 8.059 & 0.589 & 0.422 & 0.226 \\
128$\times$128 & 4.885 & 42.865 & 8.059 & 0.589 & 0.422 & 0.225 \\
256$\times$256 & 4.884 & 42.865 & 8.059 & 0.589 & 0.422 & 0.225 \\
\hline
Benchmark & 4.884 & 42.865 & 8.059 & 0.589 & 0.422 & 0.225 \\
\end{tabular}
\end{minipage}}
  \end{tabular}
\end{center}
  \label{tab:blankenbach2a-b}
\end{table*}
 


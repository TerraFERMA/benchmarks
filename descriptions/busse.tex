% Copyright (C) 2013 Columbia University in the City of New York and others.
%
% Please see the AUTHORS file in the main source directory for a full list
% of contributors.
%
% This file is part of TerraFERMA.
%
% TerraFERMA is free software: you can redistribute it and/or modify
% it under the terms of the GNU Lesser General Public License as published by
% the Free Software Foundation, either version 3 of the License, or
% (at your option) any later version.
%
% TerraFERMA is distributed in the hope that it will be useful,
% but WITHOUT ANY WARRANTY; without even the implied warranty of
% MERCHANTABILITY or FITNESS FOR A PARTICULAR PURPOSE. See the
% GNU Lesser General Public License for more details.
%
% You should have received a copy of the GNU Lesser General Public License
% along with TerraFERMA. If not, see <http://www.gnu.org/licenses/>.

\chapter{Incompressible Three-Dimensional Convection: \citeauthor{Busse1993}} \label{sec:busse}

\citet{Busse1993} described several benchmarks for
incompressible convection in a three-dimensional domain of unit height
and aspect ratio in the horizontal dimensions of $a$ and $b$.  The prognostic variables are velocity, $\vec{v}$,
pressure $p$, and temperature, $T$.

For the steady-state cases considered (1a and 2) boundary conditions for temperature are $T=0$ at the top surface, $z=1$, $T=1$ at the base, $z=0$,
with insulating (homogeneous Neumann, $\nabla T\cdot\vec{n} = 0$) side-walls.  For velocity the top surface and base have no-slip
boundary conditions applied while all vertical sides impose free-slip conditions.
Simulations are run using full time-dependence on a coarse mesh until the solution reaches an approximate steady state.  This
solution is then interpolated to a variety of higher resolutions and used as an initial guess to solve the true steady state
problem.

\paragraph{Case 1a}
Isoviscous steady-state cases are defined in a domain with aspect ratio, $a=1.0079$, $b=0.6283$,
with a Rayleigh number of $3\times10^4$ (Table \ref{tab:busse1a}).

\begin{table*}[th!]
\caption{Results from 3D, isoviscous convection benchmark case 1a \citep{Busse1993}.}
\begin{center}
\begin{tabular}{cccc}
     $N$ & $c\Delta t/\delta$  & $||\epsilon_{\phi}||$ & $\epsilon_{c}$  \\
32$\times$32 & 2.00 & 1.427786e-02 & 1.182138e-02  \\
32$\times$32 & 1.00 & 3.819159e-03 & 3.255477e-03  \\
32$\times$32 & 0.50 & 1.634865e-03 & 8.485232e-04  \\
32$\times$32 & 0.25 & 2.202831e-03 & 3.357135e-04  \\
64$\times$64 & 2.00 & 1.424202e-02 & 1.188175e-02  \\
64$\times$64 & 1.00 & 3.690917e-03 & 3.187174e-03  \\
64$\times$64 & 0.50 & 9.481209e-04 & 8.235713e-04  \\
64$\times$64 & 0.25 & 4.243532e-04 & 1.921914e-04  \\
128$\times$128 & 2.00 & 1.424087e-02 & 1.188190e-02  \\
128$\times$128 & 1.00 & 3.686779e-03 & 3.194658e-03  \\
128$\times$128 & 0.50 & 9.306871e-04 & 8.119311e-04  \\
128$\times$128 & 0.25 & 2.365556e-04 & 2.073026e-04  \\
\end{tabular}

\end{center}
  \label{tab:busse1a}
\end{table*}
%
%\paragraph{Case 2}
%
%In the variable viscosity case, the
%viscosity is temperature dependent with 
%\begin{equation}
% \mu = \exp\left[Q/(T+G) - Q/(0.5+G)\right]
%\end{equation}
%where
%\begin{gather}
%Q = [255/\log{r}] - 0.25\log{r} \\
%G = [15/\log{r}] - 0.5
%\end{gather}
%with $r = \eta(T=0)/\eta(T=1) = 20$.
%Convergence results are given in Table \ref{tab:busse2} for
%$\Ra=2\times10^4$, aspect ratio $a=b=1$.
%
%\begin{table*}[th!]
%\caption{Results from 3D, variable viscosity convection benchmark case 2 \citep{Busse1993}.}
%  \label{tab:busse2}
%\end{table*}

% Copyright (C) 2013 Columbia University in the City of New York and others.
%
% Please see the AUTHORS file in the main source directory for a full list
% of contributors.
%
% This file is part of TerraFERMA.
%
% TerraFERMA is free software: you can redistribute it and/or modify
% it under the terms of the GNU Lesser General Public License as published by
% the Free Software Foundation, either version 3 of the License, or
% (at your option) any later version.
%
% TerraFERMA is distributed in the hope that it will be useful,
% but WITHOUT ANY WARRANTY; without even the implied warranty of
% MERCHANTABILITY or FITNESS FOR A PARTICULAR PURPOSE. See the
% GNU Lesser General Public License for more details.
%
% You should have received a copy of the GNU Lesser General Public License
% along with TerraFERMA. If not, see <http://www.gnu.org/licenses/>.

\chapter{Compressible Two-Dimensional Convection: \citeauthor{KingGJI2010}} \label{sec:king}

\citet{KingGJI2010} defined a series of convection benchmarks similar to
those of 
\citet{BlankenbachGJI1989} 
but with a linearized
compressible equation of state.  Solutions are sought for velocity,
%PvK: you're using bold v in the main text for velocity.
$\vec{v}$, and perturbations in the temperature, $T$, and pressure $p$,
from prescribed reference states, $\bar{T}$ and $\bar{p}$
respectively, corresponding to a reference density $\bar{\rho}$.
% PvK: above is not strictly true - some of the participating codes used full T
% not just perturbation.
The importance of compressibility is indicated by the non-dimensional dissipation number 
$Di$.
%Several other parameters are also introduced leading to an enlarged
%parameter space involving both the Rayleigh, $\Ra$, and dissipation,
%$\Di$, numbers.  
Full details of the derivation can be found in
\citet{KingGJI2010} and \citet{SchubertCUP2001}.  A reduced subset of these
parameters is considered here for benchmarking purposes.  In all cases
the domain is a two-dimensional square of unit dimensions with
free-slip boundary conditions % and no normal flow boundary conditions specified on the
for the Stokes equations at all boundaries.  The side walls are insulating
(homogeneous Neumann, $\partial_x T = 0$) for temperature while $T=0$
is specified at the top surface.

\paragraph{Extended Boussinesq Approximation (EBA)}
In the extended Boussinesq approximation the velocity field remains
divergence free and  only the temperature equation sees the addition of terms that scale with the dissipation number.
% $r_T$, of the residual, $r$, is significantly modified:
% \begin{equation}
% r = r_{\dvv} + r_{p} + r_{T}
% \end{equation}
% where:
% \begin{align}
%   r_{\dvv} =& \int_\Omega \left(\frac{\nabla\dvv_t + \nabla\dvv_t^T}{2}\right) : 2
% \left(\frac{\nabla\dvv_i+\nabla\dvv_i^T}{2}\right) \nonumber \\
% &- \int_\Omega \nabla\cdot\dvv_t p_i - \int_{\Omega}(\dvv_t)_z T_i, \\
%   r_{p} =& \int_\Omega p_t\nabla\cdot \dvv_i, \\
%   r_{T} =& \int_\Omega T_t\left((T_i - T_n) + \Delta t\theta\dvv_\theta\cdot\nabla T_i + \Delta t(1-\theta)\dvv_\theta\cdot\nabla T_n\right)
% \nonumber \\
% &+\int_\Omega T_t\left(\Delta t\theta\Di(\dvv_\theta)_z T_i + \Delta t(1-\theta)\Di(\dvv_\theta)_z T_n \right)  \nonumber \\
% &+\int_\Omega T_t\left(\Delta t\Di(\dvv_\theta)_z T_0  - 2\Delta t\Di \left(\frac{\nabla\dvv_\theta + \nabla\dvv_\theta^T}{2}\right) : \nabla\dvv_\theta \right)  \nonumber \\
%  &+\int_\Omega \left(\frac{\Delta t\theta}{\Ra}\nabla T_t \nabla T_i + \frac{\Delta t(1-\theta)}{\Ra}\nabla T_t \nabla T_n\right),
% \end{align}
% and:
% \begin{equation}
% \dvv_\theta = \theta_{\dvv}\dvv_i + (1-\theta_{\dvv})\dvv_n
% \end{equation}
% and $T_0$ is the non-dimensionalized surface temperature.  
% The corresponding ufl code is given by:
% \begin{lstlisting}[language=Python]
% v_theta = theta_v*v_i + (1.-theta_v)*v_n
% recRa   = 1./Ra
%
% rv = (inner(sym(grad(v_t)), 2*sym(grad(v_i))) \
%       - div(v_t)*p_i - T_i*v_t[1] \
%      )*dx
% rp = p_t*div(v_i)*dx
% rT = (T_t*((T_i - T_n) \
%            + dt*theta*inner(v_theta, grad(T_i)) \
%            + dt*(1.-theta)*inner(v_theta, grad(T_n)) \
%            + dt*theta*Di*v_theta[1]*T_i \
%            + dt*(1.-theta)*Di*v_theta[1]*T_n \
%            + dt*Di*v_theta[1]*T0 \
%            - dt*Di*inner(2*sym(grad(v_theta)), grad(v_theta))) \
%       + dt*recRa*theta*inner(grad(T_t), grad(T_i)) \
%       + dt*recRa*(1.-theta)*inner(grad(T_t), grad(T_n)) \
%      )*dx
%
% r = rv + rp + rT
% \end{lstlisting}
An additional boundary condition for temperature, $T=1$ is applied at
the base of the domain.  Simulations are run at a variety of
resolutions, all structured but with refinement at the boundaries.
Convergence results for EBA and a range of $\Ra$ and $\Di$
can be found in Tables \ref{tab:KingEBARa1.e4}--\ref{tab:KingEBARa1.e5}.



\paragraph{Truncated Anelastic Liquid Approximation (TALA)}
A first approximation that includes the effects of compressibility in the 
Stokes equations, but ignoring the pressure effect on buoyancy is the
%Introducing a pressure dependence on density but ignoring the
%contribution this makes to buoyancy leads to the 
truncated anelastic liquid approximation (TALA).  
The velocity field now has a non-zero
divergence. % leading to a more substantial modification of the residual.
% $r$:
% \begin{equation}
% r = r_{\dvv} + r_{p} + r_{T}
% \end{equation}
% where:
% \begin{align}
%   r_{\dvv} =& \int_\Omega \left(\frac{\nabla\dvv_t + \nabla\dvv_t^T}{2}\right) : \left(2
% \left(\frac{\nabla\dvv_i+\nabla\dvv_i^T}{2}\right) - \frac{2}{3}\nabla\cdot\dvv_i\right) \nonumber \\
% &- \int_\Omega \nabla\cdot\dvv_t p_i + \int_{\Omega}\dvv_t\cdot\hat{\dvg}\bar{\rho}\bar{\alpha} T_i, \\
%   r_{p} =& \int_\Omega p_t\nabla\cdot\left(\bar{\rho}\dvv_i\right), \\
%   r_{T} =& \int_\Omega T_t\bar{\rho}\bar{c_p}\left((T_i - T_n) + \Delta t\theta\dvv_\theta\cdot\nabla T_i + \Delta t(1-\theta)\dvv_\theta\cdot\nabla T_n\right)
% \nonumber \\
% &-\int_\Omega T_t\left(\Delta t\theta\Di\dvv_\theta\cdot\hat{\dvg} T_i + \Delta t(1-\theta)\Di\dvv_\theta\cdot\hat{\dvg} T_n \right)  \nonumber \\
% &-\int_\Omega T_t\Delta t\Di \left(2\frac{\nabla\dvv_\theta + \nabla\dvv_\theta^T}{2} - \frac{2}{3}\nabla\cdot\dvv_\theta\right) : \nabla\dvv_\theta  \nonumber \\
%  &+\int_\Omega \left(\frac{\Delta t\theta}{\Ra}\nabla T_t \nabla T_i + \frac{\Delta t(1-\theta)}{\Ra}\nabla T_t \nabla T_n + \frac{\Delta
% t}{\Ra}\nabla T_t \nabla \bar{T}\right),
% \end{align}
% and:
% \begin{align}
% \bar{\rho} =& \exp\left(\frac{(1-z)\Di}{\gamma_r}\right) \\
% \bar{T} =& T_\text{surf}\exp\left((1-z)\Di\right) - T_\text{surf} \\
% \bar{\alpha} =& 1 \\
% \bar{c_p} =& 1 \\
% \bar{\chi} =& 1 \\
% \dvv_\theta =& \theta_{\dvv}\dvv_i + (1-\theta_{\dvv})\dvv_n.
% \end{align}
% The corresponding ufl code is given by:
% \begin{lstlisting}[language=Python]
% rhobar   = exp(depth*Di/gammar) 
% Tbar     = Tsurf*exp(Di*depth) - Tsurf 
% alphabar = 1.0
% cpbar    = 1.0
% Chibar   = 1.0
%
% v_theta  = theta_v*v_i + (1.-theta_v)*v_n
% recRa    = 1./Ra
%
% I = Identity(dim)
% def stress(v):
%   return 2*sym(grad(v))-0.666666666666666*I*div(v)
%
% rv = (inner(sym(grad(v_t)), stress(v_i)) - div(v_t)*p_i \
%       + inner(v_t, gravity)*rhobar*alphabar*T_i \
%      )*dx
% rp = p_t*div(rhobar*v_i)*dx
% rT = (T_t*(rhobar*cpbar*((T_i - T_n) \
%                          + dt*theta*inner(v_theta, grad(T_i)) \
%                          + dt*(1.-theta)*inner(v_theta, grad(T_n))) \
%            - dt*theta*Di*inner(v_theta, gravity)*alphabar*rhobar*T_i \
%            - dt*(1.-theta)*Di*inner(v_theta, gravity)*alphabar*rhobar*T_n \
%            - dt*Di*inner(stress(v_theta), grad(v_theta))) \
%       + dt*recRa*theta*inner(grad(T_t), grad(T_i)) \
%       + dt*recRa*(1.-theta)*inner(grad(T_t), grad(T_n)) \
%       + dt*recRa*inner(grad(T_t), grad(Tbar))\
%      )*dx
%
% r = rv + rp + rT
% \end{lstlisting}
Since we solve our equations in potential temperature we use a modified 
temperature condition at the lower boundary, which is now given 
%At the base of the domain the temperature boundary condition is given
by $T = 1 + T_0\left(1 - \exp(\Di)\right)$.  Simulations are run at a
variety of resolutions, all structured but with refinement at the
boundaries.  Convergence results for TALA for a range of
$\Ra$, and $\Di$  can be found in Tables \ref{tab:KingTALARa1.e4}--\ref{tab:KingTALARa1.e5}.


\paragraph{Anelastic Liquid Approximation (ALA)}
In the anelastic liquid approximation (ALA) the effects of the
pressure on buoyancy are now included. % in the velocity residual.
% , $r_{\dvv}$:
% \begin{align}
%   r_{\dvv} =& \int_\Omega \left(\frac{\nabla\dvv_t + \nabla\dvv_t^T}{2}\right) : \left(2
% \left(\frac{\nabla\dvv_i+\nabla\dvv_i^T}{2}\right) - \frac{2}{3}\nabla\cdot\dvv_i\right) \nonumber \\
% &- \int_\Omega \nabla\cdot\dvv_t p_i + \int_{\Omega}\dvv_t\cdot\hat{\dvg}\left(\bar{\rho}\bar{\alpha} T_i -
% \frac{\Di}{\gamma_r}\frac{c_{p_r}}{c_{v_r}}\bar{\rho}\bar{\chi}p_i\right) 
% \end{align}
% leading only to a small modification of the ufl code:
% \begin{lstlisting}[language=Python]
% rv = (inner(sym(grad(v_t)), stress(v_i)) - div(v_t)*p_i \
%       + inner(v_t, gravity)*(rhobar*alphabar*T_i  \
%                              - (Di/gammar)*(cpr/cvr)*rhobar*Chibar*p_i)\
%      )*dx
% \end{lstlisting}
As in TALA the temperature at the base of the domain is given by $T =
1 + T_0\left(1 - \exp(\Di)\right)$.  Simulations are run at a variety
of resolutions, all structured but with refinement at the boundaries.
Convergence results for ALA for a range of
$\Ra$, and $\Di$  can be found in Tables \ref{tab:KingALARa1.e4}--\ref{tab:kingalaRa5.e5Di0.25}.



\paragraph{Temperature Dependent Viscosity ALA}
One set of benchmarks uses the same temperature-dependence of
viscosity as in the \citet{BlankenbachGJI1989} case 2a.
%Introducing a temperature dependent viscosity also only modifies the velocity residual, $r_v$, such that:
%\begin{align}
%  r_{\dvv} =& \int_\Omega \left(\frac{\nabla\dvv_t + \nabla\dvv_t^T}{2}\right) : \left(2\mu
%\left(\frac{\nabla\dvv_i+\nabla\dvv_i^T}{2}\right) - \frac{2}{3}\nabla\cdot\dvv_i\right) \nonumber \\
%&- \int_\Omega \nabla\cdot\dvv_t p_i + \int_{\Omega}\dvv_t\cdot\hat{\dvg}\left(\bar{\rho}\bar{\alpha} T_i -
%\frac{\Di}{\gamma_r}\frac{c_{p_r}}{c_{v_r}}\bar{\rho}\bar{\chi}p_i\right)
%\end{align}
%where:
%\begin{equation}
% \mu = \exp\left(-bT_i\right).
%\end{equation}
% Leading to a small modification of the underlying ufl code:
% \begin{lstlisting}[language=Python]
% b = 6.9077552789821368
% mu = exp(-b*T_i)
%
% def stress(v):
%   return 2*mu*sym(grad(v))-0.666666666666666*I*mu*div(v)
% \end{lstlisting}
Simulations are run at a variety of resolutions, all structured but
with refinement at the boundaries.  Convergence results for  a range
of $\Ra$, and $\Di$  can be found in Table \ref{tab:KingTdepALARa1.e4}.

\begin{table*}[h]
\caption{Results from 2D, isoviscous square convection benchmark
  cases using the extended Bousinesq approximation (EBA) at
  $\Ra = 10^{4}$ \citep{KingGJI2010}}  
\begin{center}
  \begin{tabular}{ll}
\fbox{\begin{minipage}[c]{0.025\textwidth}\begin{sideways}$\Di = 0.25$\end{sideways}\end{minipage}} 
&
\fbox{\begin{minipage}[c]{0.915\textwidth}\begin{tabular}{cccc}
     $N$ & $c\Delta t/\delta$  & $||\epsilon_{\phi}||$ & $\epsilon_{c}$  \\
32$\times$32 & 2.00 & 1.427786e-02 & 1.182138e-02  \\
32$\times$32 & 1.00 & 3.819159e-03 & 3.255477e-03  \\
32$\times$32 & 0.50 & 1.634865e-03 & 8.485232e-04  \\
32$\times$32 & 0.25 & 2.202831e-03 & 3.357135e-04  \\
64$\times$64 & 2.00 & 1.424202e-02 & 1.188175e-02  \\
64$\times$64 & 1.00 & 3.690917e-03 & 3.187174e-03  \\
64$\times$64 & 0.50 & 9.481209e-04 & 8.235713e-04  \\
64$\times$64 & 0.25 & 4.243532e-04 & 1.921914e-04  \\
128$\times$128 & 2.00 & 1.424087e-02 & 1.188190e-02  \\
128$\times$128 & 1.00 & 3.686779e-03 & 3.194658e-03  \\
128$\times$128 & 0.50 & 9.306871e-04 & 8.119311e-04  \\
128$\times$128 & 0.25 & 2.365556e-04 & 2.073026e-04  \\
\end{tabular}
\end{minipage}} \\
\\
\fbox{\begin{minipage}[c]{0.025\textwidth}\begin{sideways}$\Di = 0.5$\end{sideways}\end{minipage}} 
&
\fbox{\begin{minipage}[c]{0.915\textwidth}\begin{tabular}{cccc}
     $N$ & $c\Delta t/\delta$  & $||\epsilon_{\phi}||$ & $\epsilon_{c}$  \\
32$\times$32 & 2.00 & 1.427786e-02 & 1.182138e-02  \\
32$\times$32 & 1.00 & 3.819159e-03 & 3.255477e-03  \\
32$\times$32 & 0.50 & 1.634865e-03 & 8.485232e-04  \\
32$\times$32 & 0.25 & 2.202831e-03 & 3.357135e-04  \\
64$\times$64 & 2.00 & 1.424202e-02 & 1.188175e-02  \\
64$\times$64 & 1.00 & 3.690917e-03 & 3.187174e-03  \\
64$\times$64 & 0.50 & 9.481209e-04 & 8.235713e-04  \\
64$\times$64 & 0.25 & 4.243532e-04 & 1.921914e-04  \\
128$\times$128 & 2.00 & 1.424087e-02 & 1.188190e-02  \\
128$\times$128 & 1.00 & 3.686779e-03 & 3.194658e-03  \\
128$\times$128 & 0.50 & 9.306871e-04 & 8.119311e-04  \\
128$\times$128 & 0.25 & 2.365556e-04 & 2.073026e-04  \\
\end{tabular}
\end{minipage}} \\
\\
\fbox{\begin{minipage}[c]{0.025\textwidth}\begin{sideways}$\Di = 1.0$\end{sideways}\end{minipage}} 
&
\fbox{\begin{minipage}[c]{0.915\textwidth}\begin{tabular}{cccc}
     $N$ & $c\Delta t/\delta$  & $||\epsilon_{\phi}||$ & $\epsilon_{c}$  \\
32$\times$32 & 2.00 & 1.427786e-02 & 1.182138e-02  \\
32$\times$32 & 1.00 & 3.819159e-03 & 3.255477e-03  \\
32$\times$32 & 0.50 & 1.634865e-03 & 8.485232e-04  \\
32$\times$32 & 0.25 & 2.202831e-03 & 3.357135e-04  \\
64$\times$64 & 2.00 & 1.424202e-02 & 1.188175e-02  \\
64$\times$64 & 1.00 & 3.690917e-03 & 3.187174e-03  \\
64$\times$64 & 0.50 & 9.481209e-04 & 8.235713e-04  \\
64$\times$64 & 0.25 & 4.243532e-04 & 1.921914e-04  \\
128$\times$128 & 2.00 & 1.424087e-02 & 1.188190e-02  \\
128$\times$128 & 1.00 & 3.686779e-03 & 3.194658e-03  \\
128$\times$128 & 0.50 & 9.306871e-04 & 8.119311e-04  \\
128$\times$128 & 0.25 & 2.365556e-04 & 2.073026e-04  \\
\end{tabular}
\end{minipage}}
  \end{tabular}
\end{center}
  \label{tab:KingEBARa1.e4}
\end{table*}

\begin{table*}[h]
\caption{Results from 2D, isoviscous square convection benchmark
  cases using the extended Bousinesq approximation (EBA) at
  $\Ra = 10^{5}$ \citep{KingGJI2010}}  
\begin{center}
  \begin{tabular}{ll}
\fbox{\begin{minipage}[c]{0.025\textwidth}\begin{sideways}$\Di = 0.25$\end{sideways}\end{minipage}} 
&
\fbox{\begin{minipage}[c]{0.915\textwidth}\begin{tabular}{cccc}
     $N$ & $c\Delta t/\delta$  & $||\epsilon_{\phi}||$ & $\epsilon_{c}$  \\
32$\times$32 & 2.00 & 1.427786e-02 & 1.182138e-02  \\
32$\times$32 & 1.00 & 3.819159e-03 & 3.255477e-03  \\
32$\times$32 & 0.50 & 1.634865e-03 & 8.485232e-04  \\
32$\times$32 & 0.25 & 2.202831e-03 & 3.357135e-04  \\
64$\times$64 & 2.00 & 1.424202e-02 & 1.188175e-02  \\
64$\times$64 & 1.00 & 3.690917e-03 & 3.187174e-03  \\
64$\times$64 & 0.50 & 9.481209e-04 & 8.235713e-04  \\
64$\times$64 & 0.25 & 4.243532e-04 & 1.921914e-04  \\
128$\times$128 & 2.00 & 1.424087e-02 & 1.188190e-02  \\
128$\times$128 & 1.00 & 3.686779e-03 & 3.194658e-03  \\
128$\times$128 & 0.50 & 9.306871e-04 & 8.119311e-04  \\
128$\times$128 & 0.25 & 2.365556e-04 & 2.073026e-04  \\
\end{tabular}
\end{minipage}} \\
\\
\fbox{\begin{minipage}[c]{0.025\textwidth}\begin{sideways}$\Di = 0.5$\end{sideways}\end{minipage}} 
&
\fbox{\begin{minipage}[c]{0.915\textwidth}\begin{tabular}{cccc}
     $N$ & $c\Delta t/\delta$  & $||\epsilon_{\phi}||$ & $\epsilon_{c}$  \\
32$\times$32 & 2.00 & 1.427786e-02 & 1.182138e-02  \\
32$\times$32 & 1.00 & 3.819159e-03 & 3.255477e-03  \\
32$\times$32 & 0.50 & 1.634865e-03 & 8.485232e-04  \\
32$\times$32 & 0.25 & 2.202831e-03 & 3.357135e-04  \\
64$\times$64 & 2.00 & 1.424202e-02 & 1.188175e-02  \\
64$\times$64 & 1.00 & 3.690917e-03 & 3.187174e-03  \\
64$\times$64 & 0.50 & 9.481209e-04 & 8.235713e-04  \\
64$\times$64 & 0.25 & 4.243532e-04 & 1.921914e-04  \\
128$\times$128 & 2.00 & 1.424087e-02 & 1.188190e-02  \\
128$\times$128 & 1.00 & 3.686779e-03 & 3.194658e-03  \\
128$\times$128 & 0.50 & 9.306871e-04 & 8.119311e-04  \\
128$\times$128 & 0.25 & 2.365556e-04 & 2.073026e-04  \\
\end{tabular}
\end{minipage}} \\
\\
\fbox{\begin{minipage}[c]{0.025\textwidth}\begin{sideways}$\Di = 1.0$\end{sideways}\end{minipage}} 
&
\fbox{\begin{minipage}[c]{0.915\textwidth}\begin{tabular}{cccc}
     $N$ & $c\Delta t/\delta$  & $||\epsilon_{\phi}||$ & $\epsilon_{c}$  \\
32$\times$32 & 2.00 & 1.427786e-02 & 1.182138e-02  \\
32$\times$32 & 1.00 & 3.819159e-03 & 3.255477e-03  \\
32$\times$32 & 0.50 & 1.634865e-03 & 8.485232e-04  \\
32$\times$32 & 0.25 & 2.202831e-03 & 3.357135e-04  \\
64$\times$64 & 2.00 & 1.424202e-02 & 1.188175e-02  \\
64$\times$64 & 1.00 & 3.690917e-03 & 3.187174e-03  \\
64$\times$64 & 0.50 & 9.481209e-04 & 8.235713e-04  \\
64$\times$64 & 0.25 & 4.243532e-04 & 1.921914e-04  \\
128$\times$128 & 2.00 & 1.424087e-02 & 1.188190e-02  \\
128$\times$128 & 1.00 & 3.686779e-03 & 3.194658e-03  \\
128$\times$128 & 0.50 & 9.306871e-04 & 8.119311e-04  \\
128$\times$128 & 0.25 & 2.365556e-04 & 2.073026e-04  \\
\end{tabular}
\end{minipage}}
  \end{tabular}
\end{center}
  \label{tab:KingEBARa1.e5}
\end{table*}



\begin{table*}[h]
\caption{Results from 2D, isoviscous square convection benchmark
  cases using the truncated anelastic approximation (TALA) at $\Ra = 10^{4}$  \citep{KingGJI2010}}  
\begin{center}
  \begin{tabular}{ll}
\fbox{\begin{minipage}[c]{0.025\textwidth}\begin{sideways}$\Di = 0.25$\end{sideways}\end{minipage}} 
&
\fbox{\begin{minipage}[c]{0.915\textwidth}\begin{tabular}{cccc}
     $N$ & $c\Delta t/\delta$  & $||\epsilon_{\phi}||$ & $\epsilon_{c}$  \\
32$\times$32 & 2.00 & 1.427786e-02 & 1.182138e-02  \\
32$\times$32 & 1.00 & 3.819159e-03 & 3.255477e-03  \\
32$\times$32 & 0.50 & 1.634865e-03 & 8.485232e-04  \\
32$\times$32 & 0.25 & 2.202831e-03 & 3.357135e-04  \\
64$\times$64 & 2.00 & 1.424202e-02 & 1.188175e-02  \\
64$\times$64 & 1.00 & 3.690917e-03 & 3.187174e-03  \\
64$\times$64 & 0.50 & 9.481209e-04 & 8.235713e-04  \\
64$\times$64 & 0.25 & 4.243532e-04 & 1.921914e-04  \\
128$\times$128 & 2.00 & 1.424087e-02 & 1.188190e-02  \\
128$\times$128 & 1.00 & 3.686779e-03 & 3.194658e-03  \\
128$\times$128 & 0.50 & 9.306871e-04 & 8.119311e-04  \\
128$\times$128 & 0.25 & 2.365556e-04 & 2.073026e-04  \\
\end{tabular}
\end{minipage}} \\
\\
\fbox{\begin{minipage}[c]{0.025\textwidth}\begin{sideways}$\Di = 1.0$\end{sideways}\end{minipage}} 
&
\fbox{\begin{minipage}[c]{0.915\textwidth}\begin{tabular}{cccc}
     $N$ & $c\Delta t/\delta$  & $||\epsilon_{\phi}||$ & $\epsilon_{c}$  \\
32$\times$32 & 2.00 & 1.427786e-02 & 1.182138e-02  \\
32$\times$32 & 1.00 & 3.819159e-03 & 3.255477e-03  \\
32$\times$32 & 0.50 & 1.634865e-03 & 8.485232e-04  \\
32$\times$32 & 0.25 & 2.202831e-03 & 3.357135e-04  \\
64$\times$64 & 2.00 & 1.424202e-02 & 1.188175e-02  \\
64$\times$64 & 1.00 & 3.690917e-03 & 3.187174e-03  \\
64$\times$64 & 0.50 & 9.481209e-04 & 8.235713e-04  \\
64$\times$64 & 0.25 & 4.243532e-04 & 1.921914e-04  \\
128$\times$128 & 2.00 & 1.424087e-02 & 1.188190e-02  \\
128$\times$128 & 1.00 & 3.686779e-03 & 3.194658e-03  \\
128$\times$128 & 0.50 & 9.306871e-04 & 8.119311e-04  \\
128$\times$128 & 0.25 & 2.365556e-04 & 2.073026e-04  \\
\end{tabular}
\end{minipage}} \\
\\
\fbox{\begin{minipage}[c]{0.025\textwidth}\begin{sideways}$\Di = 1.5$\end{sideways}\end{minipage}} 
&
\fbox{\begin{minipage}[c]{0.915\textwidth}\begin{tabular}{cccc}
     $N$ & $c\Delta t/\delta$  & $||\epsilon_{\phi}||$ & $\epsilon_{c}$  \\
32$\times$32 & 2.00 & 1.427786e-02 & 1.182138e-02  \\
32$\times$32 & 1.00 & 3.819159e-03 & 3.255477e-03  \\
32$\times$32 & 0.50 & 1.634865e-03 & 8.485232e-04  \\
32$\times$32 & 0.25 & 2.202831e-03 & 3.357135e-04  \\
64$\times$64 & 2.00 & 1.424202e-02 & 1.188175e-02  \\
64$\times$64 & 1.00 & 3.690917e-03 & 3.187174e-03  \\
64$\times$64 & 0.50 & 9.481209e-04 & 8.235713e-04  \\
64$\times$64 & 0.25 & 4.243532e-04 & 1.921914e-04  \\
128$\times$128 & 2.00 & 1.424087e-02 & 1.188190e-02  \\
128$\times$128 & 1.00 & 3.686779e-03 & 3.194658e-03  \\
128$\times$128 & 0.50 & 9.306871e-04 & 8.119311e-04  \\
128$\times$128 & 0.25 & 2.365556e-04 & 2.073026e-04  \\
\end{tabular}
\end{minipage}}
  \end{tabular}
\end{center}
  \label{tab:KingTALARa1.e4}
\end{table*}

\begin{table*}[h]
\caption{Results from 2D, isoviscous square convection benchmark
  cases using the truncated anelastic approximation (TALA) at $\Ra = 10^{5}$ \citep{KingGJI2010}}  
\begin{center}
  \begin{tabular}{ll}
\fbox{\begin{minipage}[c]{0.025\textwidth}\begin{sideways}$\Di = 0.25$\end{sideways}\end{minipage}} 
&
\fbox{\begin{minipage}[c]{0.915\textwidth}\begin{tabular}{cccc}
     $N$ & $c\Delta t/\delta$  & $||\epsilon_{\phi}||$ & $\epsilon_{c}$  \\
32$\times$32 & 2.00 & 1.427786e-02 & 1.182138e-02  \\
32$\times$32 & 1.00 & 3.819159e-03 & 3.255477e-03  \\
32$\times$32 & 0.50 & 1.634865e-03 & 8.485232e-04  \\
32$\times$32 & 0.25 & 2.202831e-03 & 3.357135e-04  \\
64$\times$64 & 2.00 & 1.424202e-02 & 1.188175e-02  \\
64$\times$64 & 1.00 & 3.690917e-03 & 3.187174e-03  \\
64$\times$64 & 0.50 & 9.481209e-04 & 8.235713e-04  \\
64$\times$64 & 0.25 & 4.243532e-04 & 1.921914e-04  \\
128$\times$128 & 2.00 & 1.424087e-02 & 1.188190e-02  \\
128$\times$128 & 1.00 & 3.686779e-03 & 3.194658e-03  \\
128$\times$128 & 0.50 & 9.306871e-04 & 8.119311e-04  \\
128$\times$128 & 0.25 & 2.365556e-04 & 2.073026e-04  \\
\end{tabular}
\end{minipage}} \\
\\
\fbox{\begin{minipage}[c]{0.025\textwidth}\begin{sideways}$\Di = 1.0$\end{sideways}\end{minipage}} 
&
\fbox{\begin{minipage}[c]{0.915\textwidth}\begin{tabular}{cccc}
     $N$ & $c\Delta t/\delta$  & $||\epsilon_{\phi}||$ & $\epsilon_{c}$  \\
32$\times$32 & 2.00 & 1.427786e-02 & 1.182138e-02  \\
32$\times$32 & 1.00 & 3.819159e-03 & 3.255477e-03  \\
32$\times$32 & 0.50 & 1.634865e-03 & 8.485232e-04  \\
32$\times$32 & 0.25 & 2.202831e-03 & 3.357135e-04  \\
64$\times$64 & 2.00 & 1.424202e-02 & 1.188175e-02  \\
64$\times$64 & 1.00 & 3.690917e-03 & 3.187174e-03  \\
64$\times$64 & 0.50 & 9.481209e-04 & 8.235713e-04  \\
64$\times$64 & 0.25 & 4.243532e-04 & 1.921914e-04  \\
128$\times$128 & 2.00 & 1.424087e-02 & 1.188190e-02  \\
128$\times$128 & 1.00 & 3.686779e-03 & 3.194658e-03  \\
128$\times$128 & 0.50 & 9.306871e-04 & 8.119311e-04  \\
128$\times$128 & 0.25 & 2.365556e-04 & 2.073026e-04  \\
\end{tabular}
\end{minipage}} \\
\\
\fbox{\begin{minipage}[c]{0.025\textwidth}\begin{sideways}$\Di = 1.5$\end{sideways}\end{minipage}} 
&
\fbox{\begin{minipage}[c]{0.915\textwidth}\begin{tabular}{cccc}
     $N$ & $c\Delta t/\delta$  & $||\epsilon_{\phi}||$ & $\epsilon_{c}$  \\
32$\times$32 & 2.00 & 1.427786e-02 & 1.182138e-02  \\
32$\times$32 & 1.00 & 3.819159e-03 & 3.255477e-03  \\
32$\times$32 & 0.50 & 1.634865e-03 & 8.485232e-04  \\
32$\times$32 & 0.25 & 2.202831e-03 & 3.357135e-04  \\
64$\times$64 & 2.00 & 1.424202e-02 & 1.188175e-02  \\
64$\times$64 & 1.00 & 3.690917e-03 & 3.187174e-03  \\
64$\times$64 & 0.50 & 9.481209e-04 & 8.235713e-04  \\
64$\times$64 & 0.25 & 4.243532e-04 & 1.921914e-04  \\
128$\times$128 & 2.00 & 1.424087e-02 & 1.188190e-02  \\
128$\times$128 & 1.00 & 3.686779e-03 & 3.194658e-03  \\
128$\times$128 & 0.50 & 9.306871e-04 & 8.119311e-04  \\
128$\times$128 & 0.25 & 2.365556e-04 & 2.073026e-04  \\
\end{tabular}
\end{minipage}}
  \end{tabular}
\end{center}
  \label{tab:KingTALARa1.e5}
\end{table*}

\begin{table*}[h]
\caption{Results from 2D, isoviscous square convection benchmark
  cases using the anelastic liquid approximation (ALA) at $\Ra = 10^{4}$  \citep{KingGJI2010}}  
\begin{center}
  \begin{tabular}{ll}
\fbox{\begin{minipage}[c]{0.025\textwidth}\begin{sideways}$\Di = 0.25$\end{sideways}\end{minipage}} 
&
\fbox{\begin{minipage}[c]{0.915\textwidth}\begin{tabular}{cccc}
     $N$ & $c\Delta t/\delta$  & $||\epsilon_{\phi}||$ & $\epsilon_{c}$  \\
32$\times$32 & 2.00 & 1.427786e-02 & 1.182138e-02  \\
32$\times$32 & 1.00 & 3.819159e-03 & 3.255477e-03  \\
32$\times$32 & 0.50 & 1.634865e-03 & 8.485232e-04  \\
32$\times$32 & 0.25 & 2.202831e-03 & 3.357135e-04  \\
64$\times$64 & 2.00 & 1.424202e-02 & 1.188175e-02  \\
64$\times$64 & 1.00 & 3.690917e-03 & 3.187174e-03  \\
64$\times$64 & 0.50 & 9.481209e-04 & 8.235713e-04  \\
64$\times$64 & 0.25 & 4.243532e-04 & 1.921914e-04  \\
128$\times$128 & 2.00 & 1.424087e-02 & 1.188190e-02  \\
128$\times$128 & 1.00 & 3.686779e-03 & 3.194658e-03  \\
128$\times$128 & 0.50 & 9.306871e-04 & 8.119311e-04  \\
128$\times$128 & 0.25 & 2.365556e-04 & 2.073026e-04  \\
\end{tabular}
\end{minipage}} \\
\\
\fbox{\begin{minipage}[c]{0.025\textwidth}\begin{sideways}$\Di = 1.0$\end{sideways}\end{minipage}} 
&
\fbox{\begin{minipage}[c]{0.915\textwidth}\begin{tabular}{cccc}
     $N$ & $c\Delta t/\delta$  & $||\epsilon_{\phi}||$ & $\epsilon_{c}$  \\
32$\times$32 & 2.00 & 1.427786e-02 & 1.182138e-02  \\
32$\times$32 & 1.00 & 3.819159e-03 & 3.255477e-03  \\
32$\times$32 & 0.50 & 1.634865e-03 & 8.485232e-04  \\
32$\times$32 & 0.25 & 2.202831e-03 & 3.357135e-04  \\
64$\times$64 & 2.00 & 1.424202e-02 & 1.188175e-02  \\
64$\times$64 & 1.00 & 3.690917e-03 & 3.187174e-03  \\
64$\times$64 & 0.50 & 9.481209e-04 & 8.235713e-04  \\
64$\times$64 & 0.25 & 4.243532e-04 & 1.921914e-04  \\
128$\times$128 & 2.00 & 1.424087e-02 & 1.188190e-02  \\
128$\times$128 & 1.00 & 3.686779e-03 & 3.194658e-03  \\
128$\times$128 & 0.50 & 9.306871e-04 & 8.119311e-04  \\
128$\times$128 & 0.25 & 2.365556e-04 & 2.073026e-04  \\
\end{tabular}
\end{minipage}} \\
\\
\fbox{\begin{minipage}[c]{0.025\textwidth}\begin{sideways}$\Di = 1.5$\end{sideways}\end{minipage}} 
&
\fbox{\begin{minipage}[c]{0.915\textwidth}\begin{tabular}{cccc}
     $N$ & $c\Delta t/\delta$  & $||\epsilon_{\phi}||$ & $\epsilon_{c}$  \\
32$\times$32 & 2.00 & 1.427786e-02 & 1.182138e-02  \\
32$\times$32 & 1.00 & 3.819159e-03 & 3.255477e-03  \\
32$\times$32 & 0.50 & 1.634865e-03 & 8.485232e-04  \\
32$\times$32 & 0.25 & 2.202831e-03 & 3.357135e-04  \\
64$\times$64 & 2.00 & 1.424202e-02 & 1.188175e-02  \\
64$\times$64 & 1.00 & 3.690917e-03 & 3.187174e-03  \\
64$\times$64 & 0.50 & 9.481209e-04 & 8.235713e-04  \\
64$\times$64 & 0.25 & 4.243532e-04 & 1.921914e-04  \\
128$\times$128 & 2.00 & 1.424087e-02 & 1.188190e-02  \\
128$\times$128 & 1.00 & 3.686779e-03 & 3.194658e-03  \\
128$\times$128 & 0.50 & 9.306871e-04 & 8.119311e-04  \\
128$\times$128 & 0.25 & 2.365556e-04 & 2.073026e-04  \\
\end{tabular}
\end{minipage}}
  \end{tabular}
\end{center}
  \label{tab:KingALARa1.e4}
\end{table*}

\begin{table*}[h]
\caption{Results from 2D, isoviscous square convection benchmark
  cases using the anelastic liquid approximation (ALA) at $\Ra = 10^{5}$  \citep{KingGJI2010}}  
\begin{center}
  \begin{tabular}{ll}
\fbox{\begin{minipage}[c]{0.025\textwidth}\begin{sideways}$\Di = 0.25$\end{sideways}\end{minipage}} 
&
\fbox{\begin{minipage}[c]{0.915\textwidth}\begin{tabular}{cccc}
     $N$ & $c\Delta t/\delta$  & $||\epsilon_{\phi}||$ & $\epsilon_{c}$  \\
32$\times$32 & 2.00 & 1.427786e-02 & 1.182138e-02  \\
32$\times$32 & 1.00 & 3.819159e-03 & 3.255477e-03  \\
32$\times$32 & 0.50 & 1.634865e-03 & 8.485232e-04  \\
32$\times$32 & 0.25 & 2.202831e-03 & 3.357135e-04  \\
64$\times$64 & 2.00 & 1.424202e-02 & 1.188175e-02  \\
64$\times$64 & 1.00 & 3.690917e-03 & 3.187174e-03  \\
64$\times$64 & 0.50 & 9.481209e-04 & 8.235713e-04  \\
64$\times$64 & 0.25 & 4.243532e-04 & 1.921914e-04  \\
128$\times$128 & 2.00 & 1.424087e-02 & 1.188190e-02  \\
128$\times$128 & 1.00 & 3.686779e-03 & 3.194658e-03  \\
128$\times$128 & 0.50 & 9.306871e-04 & 8.119311e-04  \\
128$\times$128 & 0.25 & 2.365556e-04 & 2.073026e-04  \\
\end{tabular}
\end{minipage}} \\
\\
\fbox{\begin{minipage}[c]{0.025\textwidth}\begin{sideways}$\Di = 1.0$\end{sideways}\end{minipage}} 
&
\fbox{\begin{minipage}[c]{0.915\textwidth}\begin{tabular}{cccc}
     $N$ & $c\Delta t/\delta$  & $||\epsilon_{\phi}||$ & $\epsilon_{c}$  \\
32$\times$32 & 2.00 & 1.427786e-02 & 1.182138e-02  \\
32$\times$32 & 1.00 & 3.819159e-03 & 3.255477e-03  \\
32$\times$32 & 0.50 & 1.634865e-03 & 8.485232e-04  \\
32$\times$32 & 0.25 & 2.202831e-03 & 3.357135e-04  \\
64$\times$64 & 2.00 & 1.424202e-02 & 1.188175e-02  \\
64$\times$64 & 1.00 & 3.690917e-03 & 3.187174e-03  \\
64$\times$64 & 0.50 & 9.481209e-04 & 8.235713e-04  \\
64$\times$64 & 0.25 & 4.243532e-04 & 1.921914e-04  \\
128$\times$128 & 2.00 & 1.424087e-02 & 1.188190e-02  \\
128$\times$128 & 1.00 & 3.686779e-03 & 3.194658e-03  \\
128$\times$128 & 0.50 & 9.306871e-04 & 8.119311e-04  \\
128$\times$128 & 0.25 & 2.365556e-04 & 2.073026e-04  \\
\end{tabular}
\end{minipage}} \\
\\
\fbox{\begin{minipage}[c]{0.025\textwidth}\begin{sideways}$\Di = 1.5$\end{sideways}\end{minipage}} 
&
\fbox{\begin{minipage}[c]{0.915\textwidth}\begin{tabular}{cccc}
     $N$ & $c\Delta t/\delta$  & $||\epsilon_{\phi}||$ & $\epsilon_{c}$  \\
32$\times$32 & 2.00 & 1.427786e-02 & 1.182138e-02  \\
32$\times$32 & 1.00 & 3.819159e-03 & 3.255477e-03  \\
32$\times$32 & 0.50 & 1.634865e-03 & 8.485232e-04  \\
32$\times$32 & 0.25 & 2.202831e-03 & 3.357135e-04  \\
64$\times$64 & 2.00 & 1.424202e-02 & 1.188175e-02  \\
64$\times$64 & 1.00 & 3.690917e-03 & 3.187174e-03  \\
64$\times$64 & 0.50 & 9.481209e-04 & 8.235713e-04  \\
64$\times$64 & 0.25 & 4.243532e-04 & 1.921914e-04  \\
128$\times$128 & 2.00 & 1.424087e-02 & 1.188190e-02  \\
128$\times$128 & 1.00 & 3.686779e-03 & 3.194658e-03  \\
128$\times$128 & 0.50 & 9.306871e-04 & 8.119311e-04  \\
128$\times$128 & 0.25 & 2.365556e-04 & 2.073026e-04  \\
\end{tabular}
\end{minipage}}
  \end{tabular}
\end{center}
  \label{tab:KingALARa1.e5}
\end{table*}

\begin{table*}[h]
\caption{Results from 2D, isoviscous square convection benchmark
  cases using the anelastic approximation (ALA) at $\Ra = 5
\times 10^{5}$ \citep{KingGJI2010}.} 
\begin{center}
  \begin{tabular}{ll}
\fbox{\begin{minipage}[c]{0.025\textwidth}\begin{sideways}$\Di = 0.25$\end{sideways}\end{minipage}} 
&
\fbox{\begin{minipage}[c]{0.925\textwidth}\begin{tabular}{cccc}
     $N$ & $c\Delta t/\delta$  & $||\epsilon_{\phi}||$ & $\epsilon_{c}$  \\
32$\times$32 & 2.00 & 1.427786e-02 & 1.182138e-02  \\
32$\times$32 & 1.00 & 3.819159e-03 & 3.255477e-03  \\
32$\times$32 & 0.50 & 1.634865e-03 & 8.485232e-04  \\
32$\times$32 & 0.25 & 2.202831e-03 & 3.357135e-04  \\
64$\times$64 & 2.00 & 1.424202e-02 & 1.188175e-02  \\
64$\times$64 & 1.00 & 3.690917e-03 & 3.187174e-03  \\
64$\times$64 & 0.50 & 9.481209e-04 & 8.235713e-04  \\
64$\times$64 & 0.25 & 4.243532e-04 & 1.921914e-04  \\
128$\times$128 & 2.00 & 1.424087e-02 & 1.188190e-02  \\
128$\times$128 & 1.00 & 3.686779e-03 & 3.194658e-03  \\
128$\times$128 & 0.50 & 9.306871e-04 & 8.119311e-04  \\
128$\times$128 & 0.25 & 2.365556e-04 & 2.073026e-04  \\
\end{tabular}
\end{minipage}}
  \end{tabular}
\end{center}
\label{tab:kingalaRa5.e5Di0.25}
\end{table*}


\begin{table*}[h]
\caption{Results from 2D  
square convection benchmark
  cases with temperature-dependent rheology 
using the anelastic liquid approximation ($T$
    dependent  ALA) at $\Ra = 10^{4}$  \citep{KingGJI2010}}  
\begin{center}
  \begin{tabular}{ll}
\fbox{\begin{minipage}[c]{0.025\textwidth}\begin{sideways}$\Di = 0.25$\end{sideways}\end{minipage}} 
&
\fbox{\begin{minipage}[c]{0.915\textwidth}\begin{tabular}{cccc}
     $N$ & $c\Delta t/\delta$  & $||\epsilon_{\phi}||$ & $\epsilon_{c}$  \\
32$\times$32 & 2.00 & 1.427786e-02 & 1.182138e-02  \\
32$\times$32 & 1.00 & 3.819159e-03 & 3.255477e-03  \\
32$\times$32 & 0.50 & 1.634865e-03 & 8.485232e-04  \\
32$\times$32 & 0.25 & 2.202831e-03 & 3.357135e-04  \\
64$\times$64 & 2.00 & 1.424202e-02 & 1.188175e-02  \\
64$\times$64 & 1.00 & 3.690917e-03 & 3.187174e-03  \\
64$\times$64 & 0.50 & 9.481209e-04 & 8.235713e-04  \\
64$\times$64 & 0.25 & 4.243532e-04 & 1.921914e-04  \\
128$\times$128 & 2.00 & 1.424087e-02 & 1.188190e-02  \\
128$\times$128 & 1.00 & 3.686779e-03 & 3.194658e-03  \\
128$\times$128 & 0.50 & 9.306871e-04 & 8.119311e-04  \\
128$\times$128 & 0.25 & 2.365556e-04 & 2.073026e-04  \\
\end{tabular}
\end{minipage}} \\
\\
\fbox{\begin{minipage}[c]{0.025\textwidth}\begin{sideways}$\Di = 1.0$\end{sideways}\end{minipage}} 
&
\fbox{\begin{minipage}[c]{0.915\textwidth}\begin{tabular}{cccc}
     $N$ & $c\Delta t/\delta$  & $||\epsilon_{\phi}||$ & $\epsilon_{c}$  \\
32$\times$32 & 2.00 & 1.427786e-02 & 1.182138e-02  \\
32$\times$32 & 1.00 & 3.819159e-03 & 3.255477e-03  \\
32$\times$32 & 0.50 & 1.634865e-03 & 8.485232e-04  \\
32$\times$32 & 0.25 & 2.202831e-03 & 3.357135e-04  \\
64$\times$64 & 2.00 & 1.424202e-02 & 1.188175e-02  \\
64$\times$64 & 1.00 & 3.690917e-03 & 3.187174e-03  \\
64$\times$64 & 0.50 & 9.481209e-04 & 8.235713e-04  \\
64$\times$64 & 0.25 & 4.243532e-04 & 1.921914e-04  \\
128$\times$128 & 2.00 & 1.424087e-02 & 1.188190e-02  \\
128$\times$128 & 1.00 & 3.686779e-03 & 3.194658e-03  \\
128$\times$128 & 0.50 & 9.306871e-04 & 8.119311e-04  \\
128$\times$128 & 0.25 & 2.365556e-04 & 2.073026e-04  \\
\end{tabular}
\end{minipage}} 
  \end{tabular}
\end{center}
  \label{tab:KingTdepALARa1.e4}
\end{table*}

% \paragraph{Time Dependent ALA}
\clearpage


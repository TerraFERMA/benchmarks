% Copyright (C) 2013 Columbia University in the City of New York and others.
%
% Please see the AUTHORS file in the main source directory for a full list
% of contributors.
%
% This file is part of TerraFERMA.
%
% TerraFERMA is free software: you can redistribute it and/or modify
% it under the terms of the GNU Lesser General Public License as published by
% the Free Software Foundation, either version 3 of the License, or
% (at your option) any later version.
%
% TerraFERMA is distributed in the hope that it will be useful,
% but WITHOUT ANY WARRANTY; without even the implied warranty of
% MERCHANTABILITY or FITNESS FOR A PARTICULAR PURPOSE. See the
% GNU Lesser General Public License for more details.
%
% You should have received a copy of the GNU Lesser General Public License
% along with TerraFERMA. If not, see <http://www.gnu.org/licenses/>.


\chapter{Idealized Kinematic Subduction Zones: \citeauthor{VanKekenPEPI2008}} \label{sec:vankeken}

\citet{VanKekenPEPI2008} defines a series of benchmarks for subduction
zone thermal structure using a kinematic slab and dynamic wedge. In
the simplest case (1a, Table \ref{tab:vanKeken1a-c}) of isoviscous rheology the velocity can be
described analytically and only the heat equation needs to be
solved. Subsequent cases explore the solution of the Stokes equations
for isoviscous flow (cases 1b and 1c, Table \ref{tab:vanKeken1a-c}) and the effects of
temperature-dependent (case 2a, Table \ref{tab:vanKeken2a-b}) and temperature- and stress-dependent
rheology (case 2b, Table \ref{tab:vanKeken2a-b}).

% \begin{equation}
%   r = \int_\Omega T_t\dvv_i\cdot\nabla T_i + \int_\Omega \nabla T_t \kappa' \nabla T_i,
% \end{equation}
% where:
% \begin{align}
% \kappa' =& \frac{\kappa}{d v_0}, \\
% \kappa =& 0.7272\times10^{-6}, \\
% d =& 1000, \\
% v_0 =& \frac{0.05}{365\times24\times60\times60}
% \end{align}

% \begin{lstlisting}[language=Python]
% d     = 1000.0
% v0    = 0.05/(365.*24.*60.*60.)
% kappa = 0.7272e-6
% kappaprime = kappa/(d*v0)

% bT = T_t*inner(v_i, grad(T_i)) \
%      + inner(grad(T_t), kappaprime*grad(T_i))

% r  = bT*dx_crust
% r += bT*dx_wedge
% r += bT*dx_mantle
% \end{lstlisting}

%\paragraph{Case 1b--c} Full isoviscous stokes solution in the wedge
%plus advection-diffusion of temperature. Case 1b has dirichlet
%velocity conditions on the wedge inflow boundary set to the analytic
%corner flow solution.  Case 1c, has a natural inflow boundary
%condition on the left.

% \begin{equation}
% r = r_{\dvv} + r_{p} + r_{T}
% \end{equation}
% where:
% \begin{align}
%   r_{\dvv} =& \int_{\Omega_{\text{wedge}}} \left(\frac{\nabla\dvv_t + \nabla\dvv_t^T}{2}\right) : 2
% \mu'\left(\frac{\nabla\dvv_i+\nabla\dvv_i^T}{2}\right) - \int_{\Omega_{\text{wedge}}} \nabla\cdot\dvv_t p_i, \\
%   r_{p} =& \int_{\Omega_{\text{wedge}}} p_t\nabla\cdot \dvv_i, \\
%   r_{T} =& \int_\Omega T_t\dvv_i\cdot\nabla T_i + \int_\Omega \nabla T_t \kappa' \nabla T_i,
% \end{align}
% and:
% \begin{equation}
% \mu' = 1
% \end{equation}

% \begin{lstlisting}[language=Python]
% d     = 1000.0
% v0    = 0.05/(365.*24.*60.*60.)
% kappa = 0.7272e-6
% kappaprime = kappa/(d*v0)

% muprime = 1.0

% rv  = (inner(sym(grad(v_t)), 2*muprime*sym(grad(v_i))) \
%       - div(v_t)*p_i)*dx_wedge

% rp  = p_t*div(v_i)*dx_wedge

% rT  = inner(grad(T_t), kappaprime*grad(T_i))*dx_crust
% rT += (T_t*inner(v_i,  grad(T_i)) \
%        + inner(grad(T_t), kappaprime*grad(T_i)))*dx_wedge
% rT += (T_t*inner(vp_i, grad(T_i)) \
%        + inner(grad(T_t), kappaprime*grad(T_i)))*dx_mantle

% r = rv + rp + rT
% \end{lstlisting}

%\paragraph{Case 2a} Temperature dependent viscosity stokes
%solution in the wedge with effective viscosity
%\begin{equation}
%\mu' = \frac{\mu_{\text{eff}}}{\mu_{\text{scale}}} 
%\end{equation}
%where:
%\begin{align}
%\mu_{\text{scale}} =& 1\times10^{21}, \\
%\mu_{\text{eff}} =& \frac{\mu_{\text{max}}\mu_{\text{diff}}}{\mu_{\text{max}} + \mu_{\text{diff}}}, \\
%\mu_{\text{diff}} =& A_{\text{diff}}\exp\left(\frac{E_{\text{diff}}}{RT_i}\right), \\
%R =& 8.3145, \\
%E_{\text{diff}} =& 335\times10^3, \\
%A_{\text{diff}} =& 1.32043\times10^9, \\
%\mu_{\text{max}} =& 1\times10^{26}
%\end{align}
%% \begin{lstlisting}[language=Python]
%% mumax = 1.e26
%% Adiff = 1.32043e9
%% Ediff = 335.e3
%% R     = 8.3145
%% mudiff = Adiff*exp(Ediff/(R*T_i))
%% mueff  = (mumax*mudiff)/(mumax + mudiff)
%% muscale = 1.e21
%% muprime = mueff/muscale
%% \end{lstlisting}
%
%
%
%\paragraph{Case 2b} Temperature and stress dependent effective
%viscosity in the wedge with
%\begin{equation}
%\mu_{\text{eff}} = \frac{\mu_{\text{max}}\mu_{\text{disl}}}{\mu_{\text{max}} + \mu_{\text{disl}}}
%\end{equation}
%where:
%\begin{align}
%\mu_{\text{disl}} =& A_{\text{disl}}\exp\left(\frac{E_{\text{disl}}}{nRT_i}\right)\dot{\epsilon}^{\frac{1-n}{n}}, \\
%E_{\text{disl}} =& 540\times10^3, \\
%A_{\text{disl}} =& 28968.6, \\
%\strt =& \left(\frac{\etens:\etens}{2}\right)^{\frac{1}{2}}, \\
%\etens =& \frac{\nabla\dvv_i+\nabla\dvv_i^T}{2}, \\
%n =& 3.5
%\end{align}
%% \begin{lstlisting}[language=Python]
%% mumax = 1.e26
% Adisl = 28968.6
% Edisl = 540.e3
% n = 3.5
% R = 8.3145
% nexp = (1.-n)/n
% edot = sym(grad(v_i))
% eII = sqrt(0.5*inner(edot, edot))
% mudisl = Adisl*exp(Edisl/(n*R*T_i))*(eII**nexp)
% mueff  = (mumax*mudisl)/(mumax + mudisl)
% muscale = 1.e21
% muprime = mueff/muscale
% \end{lstlisting}

\begin{table*}[h]
\caption{Results from 2D, kinematic subduction model \citep{VanKekenPEPI2008}}
\begin{center}
  \begin{tabular}{ll}
\fbox{\begin{minipage}[c]{0.075\textwidth}\begin{sideways}\parbox{3.5cm}{\begin{center}Case 1a:\\analytic corner flow\end{center}}\end{sideways}\end{minipage}} 
&
\fbox{\begin{minipage}[c]{0.575\textwidth}\begin{tabular}{c|cccccc}
    & $\Nu$ & $v_{rms}$ & $q_1$ & $q_2$ & $T_{e}$ & $z_{e}$ \\
\hline
16$\times$16 & 4.897 & 42.884 & 8.062 & 0.588 & 0.423 & 0.233 \\
32$\times$32 & 4.887 & 42.865 & 8.060 & 0.589 & 0.422 & 0.226 \\
64$\times$64 & 4.885 & 42.865 & 8.059 & 0.589 & 0.422 & 0.226 \\
128$\times$128 & 4.885 & 42.865 & 8.059 & 0.589 & 0.422 & 0.225 \\
256$\times$256 & 4.884 & 42.865 & 8.059 & 0.589 & 0.422 & 0.225 \\
\hline
Benchmark & 4.884 & 42.865 & 8.059 & 0.589 & 0.422 & 0.225 \\
\end{tabular}
\end{minipage}} \\
\\
\fbox{\begin{minipage}[c]{0.075\textwidth}\begin{sideways}\parbox{3.25cm}{\begin{center}Case 1b:\\isoviscous wedge,\\prescribed BCs\end{center}}\end{sideways}\end{minipage}} 
&
\fbox{\begin{minipage}[c]{0.575\textwidth}\begin{tabular}{c|cccccc}
    & $\Nu$ & $v_{rms}$ & $q_1$ & $q_2$ & $T_{e}$ & $z_{e}$ \\
\hline
16$\times$16 & 4.897 & 42.884 & 8.062 & 0.588 & 0.423 & 0.233 \\
32$\times$32 & 4.887 & 42.865 & 8.060 & 0.589 & 0.422 & 0.226 \\
64$\times$64 & 4.885 & 42.865 & 8.059 & 0.589 & 0.422 & 0.226 \\
128$\times$128 & 4.885 & 42.865 & 8.059 & 0.589 & 0.422 & 0.225 \\
256$\times$256 & 4.884 & 42.865 & 8.059 & 0.589 & 0.422 & 0.225 \\
\hline
Benchmark & 4.884 & 42.865 & 8.059 & 0.589 & 0.422 & 0.225 \\
\end{tabular}
\end{minipage}} \\
\\
\fbox{\begin{minipage}[c]{0.075\textwidth}\begin{sideways}\parbox{3.25cm}{\begin{center}Case 1c:\\isoviscous wedge,\\free stress BCs\end{center}}\end{sideways}\end{minipage}} 
&
\fbox{\begin{minipage}[c]{0.575\textwidth}\begin{tabular}{c|cccccc}
    & $\Nu$ & $v_{rms}$ & $q_1$ & $q_2$ & $T_{e}$ & $z_{e}$ \\
\hline
16$\times$16 & 4.897 & 42.884 & 8.062 & 0.588 & 0.423 & 0.233 \\
32$\times$32 & 4.887 & 42.865 & 8.060 & 0.589 & 0.422 & 0.226 \\
64$\times$64 & 4.885 & 42.865 & 8.059 & 0.589 & 0.422 & 0.226 \\
128$\times$128 & 4.885 & 42.865 & 8.059 & 0.589 & 0.422 & 0.225 \\
256$\times$256 & 4.884 & 42.865 & 8.059 & 0.589 & 0.422 & 0.225 \\
\hline
Benchmark & 4.884 & 42.865 & 8.059 & 0.589 & 0.422 & 0.225 \\
\end{tabular}
\end{minipage}}
  \end{tabular}
\end{center}
  \label{tab:vanKeken1a-c}
\end{table*}
\pagebreak{}

\begin{table*}[h]
\caption{Results from 2D, kinematic subduction model \citep{VanKekenPEPI2008}}
\begin{center}
  \begin{tabular}{ll}
\fbox{\begin{minipage}[c]{0.05\textwidth}\begin{sideways}\parbox{2.5cm}{\begin{center}Case 2a:\\wedge $\eta(T)$\end{center}}\end{sideways}\end{minipage}} 
&
\fbox{\begin{minipage}[c]{0.575\textwidth}\begin{tabular}{c|cccccc}
    & $\Nu$ & $v_{rms}$ & $q_1$ & $q_2$ & $T_{e}$ & $z_{e}$ \\
\hline
16$\times$16 & 4.897 & 42.884 & 8.062 & 0.588 & 0.423 & 0.233 \\
32$\times$32 & 4.887 & 42.865 & 8.060 & 0.589 & 0.422 & 0.226 \\
64$\times$64 & 4.885 & 42.865 & 8.059 & 0.589 & 0.422 & 0.226 \\
128$\times$128 & 4.885 & 42.865 & 8.059 & 0.589 & 0.422 & 0.225 \\
256$\times$256 & 4.884 & 42.865 & 8.059 & 0.589 & 0.422 & 0.225 \\
\hline
Benchmark & 4.884 & 42.865 & 8.059 & 0.589 & 0.422 & 0.225 \\
\end{tabular}
\end{minipage}} \\
\\
\fbox{\begin{minipage}[c]{0.05\textwidth}\begin{sideways}\parbox{2.25cm}{\begin{center}Case 2b:\\wedge $\eta(T,\dot\epsilon)$\end{center}}\end{sideways}\end{minipage}} 
&
\fbox{\begin{minipage}[c]{0.575\textwidth}\begin{tabular}{c|cccccc}
    & $\Nu$ & $v_{rms}$ & $q_1$ & $q_2$ & $T_{e}$ & $z_{e}$ \\
\hline
16$\times$16 & 4.897 & 42.884 & 8.062 & 0.588 & 0.423 & 0.233 \\
32$\times$32 & 4.887 & 42.865 & 8.060 & 0.589 & 0.422 & 0.226 \\
64$\times$64 & 4.885 & 42.865 & 8.059 & 0.589 & 0.422 & 0.226 \\
128$\times$128 & 4.885 & 42.865 & 8.059 & 0.589 & 0.422 & 0.225 \\
256$\times$256 & 4.884 & 42.865 & 8.059 & 0.589 & 0.422 & 0.225 \\
\hline
Benchmark & 4.884 & 42.865 & 8.059 & 0.589 & 0.422 & 0.225 \\
\end{tabular}
\end{minipage}} 
  \end{tabular}
\end{center}
  \label{tab:vanKeken2a-b}
\end{table*}

